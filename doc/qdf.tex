\documentclass[letterpaper,12pt]{article}
\usepackage{helvet}
\usepackage{fancyheadings}
\usepackage{verbatim}
\usepackage{hyperref}
\pagestyle{fancy}
\usepackage{graphicx}
\setlength\textwidth{6.5in}
\setlength\textheight{8.5in}
\newtheorem{problem_statement}{Problem Statement}
\newcommand{\TBC}{\framebox{\textbf{TO BE COMPLETED}}}
\newtheorem{assumption}{Assumption}
\newcommand{\be}{\begin{enumerate}}
\newcommand{\ee}{\end{enumerate}}
\newcommand{\bi}{\begin{itemize}}
\newcommand{\ei}{\end{itemize}}
\newtheorem{notation}{Notation}
\newtheorem{definition}{Definition}
\begin{document}
\title{QDF --- a binary JSON format}
\author{Ramesh Subramonian }
\maketitle
\thispagestyle{fancy}
\lhead{}
\chead{}
\rhead{}
% \lfoot{{\small Decision Sciences Team}}
\cfoot{}
\rfoot{{\small \thepage}}

\abstract{This paper specifies the binary JSON grammar}

\section{Grammar to parse binary data}

A QDF must have one of the following types, called a {\tt jtype}
\be
\item null
\item boolean
\item number
\item string
\item array
\item object (or map)
  \ee

Given the start of a QDF, one must be able to determine its length.
\subsection{qtypes}
\label{qtypes}
\verbatiminput{qtypes}

\subsection{Notations and Definitions}
\begin{notation}
poff = parent offset, currently unused \TBC
\end{notation}

\begin{notation}
The term ``padding'' means increasing the width of a datum to be a multiple of 8,
  with any additional bytes set to 0.
\end{notation}
\begin{notation}
  \(mathrm{pad}(x)\) is the smallest number, \(y \geq x\), such that \(y\) isa
  multiple of 8
\end{notation}
\begin{definition}
An array is said to be ``uniform'' when all elements have the same qtype
\end{definition}

\section{null}
\be
\item Byte 0   contains jtype.
\item Bytes 1 --- 3  unused
\item Bytes 4 --- 7  contains poff
\item rbc size is 8 bytes 
  \ee

\section{boolean}
\be
\item Byte 0  contains jtype.
\item Bytes 1 contains qtype
\item Bytes 2 contains value (0 or 1)
\item Bytes 3  unused
\item Bytes 4 --- 7  contains poff
\item rbc size is 8 bytes 
  \ee

\section{number}
\be
\item Byte 0 contains jtype
\item Byte 1 contains qtype
\item Bytes 2 --- 3  unused
\item Bytes 4 --- 7  contains poff
\item Byte 8-15 contains value (number)
\item rbc size is 16 bytes 
  \ee
\section{date}
This is actually \verb+ typedef struct _tm_t+ specified in qtypes
\be
\item Byte 0 contains jtype
\item Byte 1 contains qtype
\item Bytes 2 --- 3  unused
\item Bytes 4 --- 7  contains poff
\item Byte 8-15 contains value (\verb+tm_t+)
\item rbc size is 16 bytes 
  \ee
\section{string}
\be
\item Byte 0 contains jtype
\item Byte 1 contains qtype, set to SC
\item Bytes 2 --- 3  unused
\item Bytes 4 --- 7  contains poff
\item Byte 8-11 contains \verb+str_len\+ length of null-terminated string \(n\)
\item Byte 12-15 contains \verb+str_size+, space allocated, \(s\)
  \be
\item \(s\) is a multiple of 8
\item \(s \geq n+1\). 
  \be
\item The reason for the +1 is to have null characater termination
\item The reason for allocationg more than we 
need is to allow in-place updates without having to move things around.
\ee
\ee
\item rbc size is 16 bytes plus \(s\)
  \ee

\section{array}
\be
\item Byte 0 contains jtype
\item Byte 1 contains qtype
\item Byte 2-3 contains width if uniform array; 0, otherwise
\item Bytes 4 --- 7  contains poff


\item Byte 8-11 contains \verb+arr_len+, number of elements resident \(n\)
\item Byte 12-15 contains \verb+arr_size+, number of elements allocated, \(s\)

\item Byte 16-19 contains size of RBC, \(l\)
\item Byte 20-23 unused
  \ee
\subsection{uniform array}
\be
\item \(l = 24 + \mathrm{pad}(s \times w)\) bytes
\ee
\subsection{mixed array}
\be
\item \(n\) offsets stored as 4-byte integers, with padding applied. Call this
  array \(O\). Then, \(O[i]\) tells us the location of the \(i\)th element of
  the array as an offset from the start of this item.
\item concatenation of the binary representation of each item. Since an item
  must be a multiple of 8, this blob is also a multiple of 8.
\ee
\section{object}
\be
\item Byte 0 contains jtype
\item Byte 1 contains qtype {\bf if} all items same type
\item Byte 2 unused
\item Byte 3 --- true => dataframe; else, false
\item Bytes 4 --- 7  contains poff

\item Byte 8-11 contains \verb+obj_len+, number of elements resident \(n\)
\item Byte 12 --- 15 contains \verb+obj_arr_len+ if a dataframe; else, 0

  Note that this can be 0 when a dataframe is being populated. But all columns
  must have the same number of elements.

\item Byte 16 --- 19 contains length of RBC, \(l\)
\item Byte 20 --- 23 contains \verb+obj_arr_size+ if a dataframe; else, 0

For a dataframe, 

\item \(4 \times n\) bytes for key offsets, padded.
\item \(4 \times n\) bytes for val offsets, padded 
\item \(8 \times n\) bytes for (hash/idx), explained in
  Section~\ref{fast_lookup}
\item \(K\) bytes representing a concatenation of all keys, padded.
  If the lengths of the keys are \(\{l_1, l_2, \ldots l_n\}\), then \(K =
  n + \sum_{i=1}^{i=n} l_i \), with a null character between strings.
\item the values 
  \ee
\subsection{Fast lookup of keys in object}
\label{fast_lookup}
To reduce the time to locate a key in an object, we store \(n\) hash/index
pairs, each of which is a 64-bit unsigned integer. The top 48 bits represent a
hash of the key and the bottom 16 bits represent the index of the key as stored.
This limit the number of keys to 65536 --- we do not think this is an
unreasonable limitation. Let us assume that an object has 3 keys, \(x, y, z\)
stored in that order. Let us assume that \(h(x) = 30, h(y) = 10, h(z) = 20\). In
that case, we store \(\{(10,1),(20,2),(30,3)\}\). Note that thee keys are
storted in ascending order. 

At run time, when we are given a key, we compute its hash. We perform a binary
search on the hash/idx array and we have of 2 outcomes
\be
\item the hash does not exist and we inform the user accordingly
\item the hash does exist. We pick up the index \(i\) and using the key
  offset, compare the \(i^{\mathrm{th}}\) key withe one provided by the user.  Once again, we have 2 outcomes.
  \be
\item If we get a match, use the value offsets and \(i\) to return the
  correesponding value.
\item If not, we know that the hash/idx structure has failed us and we perform a
  linear search of the keys until we either find what we want or report the key
  to be missing
  \ee
  \ee

\end{document}
