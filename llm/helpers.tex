\newcommand{\getq}{{\tt get\_qtype()}}
\newcommand{\getj}{{\tt get\_jtype()}}
\newcommand{\getn}{{\tt get\_length()}}
\newcommand{\getr}{{\tt get\_read\_arr\_ptr()}}
\newcommand{\getw}{{\tt get\_write\_arr\_ptr()}}
\newcommand{\getnk}{{\tt get\_num\_keys()}}
\newcommand{\chk}{{\tt chk\_qdf()}}

\section{Read only helper functions}
\label{qdf_read_helpers}
Table~\ref{tbl_qdf_read_helpers} lists helper functions that have been provided to
you. They take as an argument an immutable pointer to \qdf\ and do {\bf not}
modify the location pointed to.
\begin{table}
  \centering 
\begin{tabular}{|l|l|l|l|} \hline \hline 
  {\bf Return} & {\bf Name} & {\bf Args} \\ \hline  \hline
  bool & \chk & ( const \qdf\ * const x) \\ \hline 
  \qt\ & getq & ( const \qdf\ * const x) \\ \hline 
  \jt\ & \getj & ( const \qdf\ * const x) \\ \hline 
  UI4  & \getn & ( const \qdf\ * const x) \\ \hline 
  const char * & \getr & ( const \qdf\ * const x) \\ \hline 
  UI4  & \getnk & ( const \qdf\ * const x) \\ \hline 
\hline
\end{tabular}
\caption{List of read only helper functions}
\label{tbl_qdf_read_helpers}
\end{table}

\subsection{\chk}
Returns {\tt true} if x is syntactically valid; {\tt false}, otherwise

\subsection{\getq}
Returns \qt\  of x 

\subsection{\getj}
Returns \jt\ of x 

\subsection{\getn}
If \jt\ of x is {\tt j\_array}, then returns length of array; else, returns 0.

\subsection{\getr}
If \jt\ of x is {\tt j\_array}, then returns {\tt NULL}.
If \qt\ of x is Qerr or Q0, then returns {\tt NULL}.
Else, returns a pointer to the \(0^{th}\) element of the array. 
The data pointed to cannot be modified. 

\subsection{\getnk}
If \jt\ x is {\tt j\_object}, then returns number of keys; else, returns 0.

